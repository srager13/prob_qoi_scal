\section{Introduction}
\label{sec:intro}

% we use QoI
% we are modeling which is easier than experimentation
% we are looking at scalability, not optimizing networks

%area
%problem
Traditional approaches to studying network scalability and performance limitations have been focused on finding theoretical limits on throughput and delay.  In many applications, however, the relationship between these metrics and the effectiveness of the network is highly non-linear.  Therefore, having a framework to evaluate network scalability with respect to achievable \emph{Quality of Information} (QoI) requirements is necessary.

Additionally, while theoretical, asymptotic analysis of individual network topologies, protocols, etc. is important, such analysis lacks the ability to quickly obtain an accurate estimate of a projected network's abilities when these individual components are pieced together. Often, researchers must turn to extensive simulation or experimentation testbeds to test proposed network setups. While the increasing availability of open source simulation frameworks like ns-3 \cite{ns3} and experimental testbeds, such as IoT-LAB \cite{iot_lab_exp_platform} and Wisebed \cite{wisebed}, have reduced the barriers to run simulations and experiments, significant time to familiarize oneself with the platform and implement the desired scenario is required. Physical testbeds also require time-sharing and have a limit on the maximum number of nodes that can be tested. Finally, once a scenario is implemented, testing a range of values for more than one or two independent variables will be time and computing intensive because the number of trials that must be run can grow quite large.

As an example of a scenario in which we are interested, imagine being given the task of deploying a wireless sensor network for a particular application. Given a proposed network with a defined size, topology, parameters, and protocols, what is the level of QoI requirements it can support?  Now, consider the converse:  Given a certain QoI that is desired by users of a network, what is the maximum number of nodes that the network can support?  Which has a bigger impact on this scalability: the imposed information requirements or the strict timeliness requirements? 

Our main contribution in this paper is a novel framework that can predict scalability and performance of a network with respect to QoI requirements for answering such questions.  We explain this framework in detail in Section \ref{sec:qoi_scalability} and provide example applications in Section \ref{sec:example_applications}.  As a second contribution we extend this framework in Section \ref{sec:delay_char}, capturing the stochastic nature of query sources and destinations as well as data requirements, and show that it can be used to characterize query delays.  In both cases, we provide results from realistic implementations in the ns-3 network simulation environment.  

We also present several pieces of supporting work.  First we provide an example of an application that relies on QoI to highlight the difference in QoI and traditional metrics in Section \ref{sec:qoi_model}.  We show in Section \ref{sec:network_design} that our framework is also quite useful in quickly and easily understanding the impact of parameters and design choices, providing a secondary benefit to network designers of allowing them to compare networks and identify tradeoffs.  Finally, we show how the framework can also provide bounds on QoI capacity in some applications in Section \ref{sec:scal_feasible_qoi}.

%in contrast to discovering theoretical, asymptotic limits, a need for network designers is to quickly obtain an accurate estimate of a network's abilities.  The goal of this paper is not in optimizing network performance; instead, we show in Section \ref{sec:network_design} that this model can also be used to quickly and easily understand the impact of parameters and design choices.  This ability provides a secondary benefit to network designers of allowing them to compare networks and identify tradeoffs. 

% this assumption is unrealistic for many applications. For these reasons, we adopt \emph{Quality of Information} (QoI), which can include a number of information attributes (many of which are context-dependent), such as completeness, diversity, credibility, creation time, and timeliness, as our measure of network performance.  Specifically, in this work, we focus on satisfiability of completeness and timeliness, explained in detail in Section \ref{sec:qoi_model}.

%The second contribution differentiating this work from previous analysis is that we focus on providing a framework that can be adjusted to determine scalability and QoI satisfiability for any instance of a network.  The wide applicability and easy reuse of this framework make it easy to compare protocols, topologies, traffic models, etc., without creating extensive simulation or experimentation testbeds, which is a much more difficult task.

%Finally, in contrast to discovering theoretical, asymptotic limits, this framework seeks to quickly obtain an accurate estimate of a network's abilities.  The goal of this paper is not in optimizing network performance; instead, we show in Section \ref{sec:network_design} that this model can also be used to quickly and easily understand the impact of parameters and design choices.  This ability provides a secondary benefit to network designers of allowing them to compare networks and identify tradeoffs. 

%why not solved
%Currently, no framework exists that provides a methodology to predict scalability and performance with respect to QoI requirements.  In this paper, our contribution is to provide such a framework.%, which we do in Section \ref{sec:qoi_scalability}.  In Section \ref{sec:validation}, we prove its effectiveness by comparing it to simulations performed in the ns3 network simulation package.  We provide examples of how it is also useful in network design in Section \ref{sec:network_design}.  Finally, we also take the concept one step further in Section \ref{sec:scal_feasible_qoi} with the introduction of a \emph{scalably feasible QoI region}, which describes the maximum QoI capacity of a particular network scenario.

%insight
%contribution
%Using timely, similarity-based image collection as a motivating application, we show the application of our framework in determining realistic limitations of an actual network scenario and validate their accuracy with simulations results from testing performed with the ns3 network simulation package.  Then, we explore the impact of changing network parameters, like topology, network size, bandwidth, etc. on satisfiable QoI and scalability, answering the questions of the impact of design choices explored above.  Finally, we also take the concept one step further with the introduction of a \emph{scalably feasible QoI region}, which describes the maximum QoI capacity of a particular network scenario. 

%Imagine given the task of deploying an ad hoc network for a particular task or application.  In executing this task, it is important to first recognize the correct metrics by which the network's performance is properly measured.  Specifically, in these networks, traditional metrics of throughput, latency, jitter, etc. are no longer the only focus.  Furthermore, the implicit assumption that linear improvements in these metrics provide an equivalently linear increase in utility to users of the network is unrealistic, because the usefulness of data is highly dependent on context.  For that reason, we instead use \emph{Quality of Information} (QoI), which includes a number of data attributes  (many of which are context-dependent), such as completeness, information diversity, credibility, creation time, and timeliness, as a general measure of network performance.  
%In addition to understanding how to properly measure the network performance, effective design of an ad hoc network requires understanding the QoI and scalability limitations of a proposed deployment.  For example, we may want to answer questions like:  With a set of traffic requirements, what is the maximum number of nodes the network can support?  Or, alternatively, given a proposed network description, how much data can we support, and what is the QoI of that data?  Furthermore, the impact of design choices on these limitations is even more valuable in optimal design.  Here, we propose answers to questions like the following:  How sensitive is delay to network size for a given topology?  Or, How does increasing requirements for completeness of information affect network size when we have a strict timeliness requirement?