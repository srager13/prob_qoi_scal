\section{Network Model}
\label{sec:network_model}

We develop the framework in this paper with an aim for application in ad hoc networks, in which nodes can act as both clients and servers, issuing queries and serving responding with data.  Additionally, nodes serve as routers, forwarding traffic for the rest of the network when it lies on a given path between a source-destination pair.  

We assume that the network has an achievable channel rate, $W$, which is the minimum effective channel rate in the network including the effects of protocols and errors. Specifically, since wireless channels are often lossy, some method of automatic repeat request (ARQ) protocol is likely to be implemented in the data link layer of the network. As in \cite{arq_schemes}, given a probability of packet loss, calculating the expected number of retransmissions for each packet is straightforward. For example, from \cite{arq_schemes}, the expected number of retransmissions, $E(N)$, for a packet with a probability of loss $p_l$ is $E(N) = \frac{1}{1-p_l}$. Using this value, we assume we can use the error-free channel rate, which we will call $W^*$, and derive an \emph{effective} channel rate to use in determining expected scalability and performance:
\begin{equation}
  W = W^* \cdot (1 - p_l)
\end{equation}
Here, an estimate of the probability of losing a packet must be known. However, as we will show, this framework creates expressions that are easy to compute, so testing a range of possible loss probabilities is also feasible if an accurate estimate is unkown.

In addition to retransmission protocols, coding techniques to detect or correct errors, such as cyclic redundancy check (CRC) codes or forward error correcting (FEC) codes, respectively, can be easily included in our framework. When these schemes are used in the network, the expected data size is simply adjusted accordingly. We discuss data sizes and elaborate on this point in Section \ref{sec:qoi_scalability} since they rely on QoI definitions in Section \ref{sec:qoi_model}.

For a traffic model, we adopt a model in which each node produces queries according to a Poisson distribution with an average rate of $\lambda$ and send the query to another node chosen at random.  This model provides a general example of expected traffic with randomness that we can use to provide an example of applying our framework in Section \ref{sec:example_applications}, but the framework is in no way limited to this particular traffic scenario.  As a second example, we provide a different network application in Section \ref{sec:scal_feasible_qoi} while introducing the concept of defining QoI capacity regions.  

\subsection{Defining Scalability}

We determine the effective scalability of the network by two methods.  First, we assume that nodes have finite queues sizes, so we say that the network is no longer scalable when the expected traffic rate is greater than the expected service rate, since such a scenario will cause queues to become overloaded and drop traffic.  The second definition of scalability is related directly to QoI.  As explained in more detail in Section \ref{sec:qoi_model}, we adopt a timeliness constraint for all query responses.  When the network is no longer capable of meeting this timeliness constraint for all queries with a probability of $(1-\epsilon)$, where $\epsilon$ can be set as close to zero as desired, then it is not considered scalable.  